
%
%  $Description: Author guidelines and sample document in LaTeX 2.09$ 
%
%  $Author: ienne $
%  $Date: 1995/09/15 15:20:59 $
%  $Revision: 1.4 $
%

\documentclass[times, 10pt,twocolumn]{article} 
%\documentclass[times, 10pt]{article} 
\usepackage{latex8}
\usepackage{times}
\usepackage{epsfig}

%\documentstyle[times,art10,twocolumn,latex8]{article}

%------------------------------------------------------------------------- 
% take the % away on next line to produce the final camera-ready version 
\pagestyle{empty}

%------------------------------------------------------------------------- 
%\linespread{1.6}

\begin{document}

\title{Collaborative Development of System Architecture - a Tool for Coping with Inconsistency}

\author{Peter Henderson\\
Electronics and Computer Science \\
University of Southampton\\  
SO17 1BJ, UK\\ p.henderson@ecs.soton.ac.uk\\
% For a paper whose authors are all at the same institution, 
% omit the following lines up until the closing ``}''.
% Additional authors and addresses can be added with ``\and'', 
% just like the second author.
\and
Matthew J. Henderson\\
Mathematics and Computer Science \\ 
Berea College\\
Berea, KY 40404, USA \\ 
matthew\_henderson@berea.edu\\
}


\maketitle
\thispagestyle{empty}

\begin{abstract}
Very large systems have an architecture that is designed to allow them to evolve through a long life. Such systems are developed by teams of architects. Among the first thing the architects do is make a model of their architecture. This model constitutes the formal architecture description based on which the engineers will build the real system.

The architecture model is normally governed by a specialised metamodel whose rules determine the consistency and completeness of the description. The development of a system architecture is carried out cooperatively but independently by team members. Consequently it is quite normal for the architecture description as a whole to be both incomplete and inconsistent. The architects strive to eventually produce a complete overall (i.e. merged) description and to eliminate the inconsistencies. 

By means of an example, we show how and why the architecture model and the metamodel must co-evolve. We describe a design tool that we have developed to support this process of co-evolution. The tool allows a team of architects to detect inconsistencies in their separate and merged models. The tool tolerates inconsistencies. It produces reports of inconsistencies which become targets for removal as the architecture description evolves.

\end{abstract}



%------------------------------------------------------------------------- 
\Section{Introduction}

\noindent Very large systems have an architecture that is designed to allow them to evolve through a long life. The usual way in which such large systems are developed is by first making a model of their architecture. The language chosen for the architecture model is usually a mixture of diagrams (in UML or SysML, for example) and lots of documentation of requirements and of interfaces to the components that are to be either procured or built. The model is thus a semi-formal architecture description. 

The architecture is normally developed incrementally and independently by a team of system architects working collaboratively. While each may strive to keep their part-model of the evolving architecture consistent, there will be inconsistencies between their independent descriptions. These inconsistencies will need to be detected and resolved when, from time to time, the models of independent architects are merged.

The more formal the architecture description, the more likely we are to be able to determine incompletenesses and inconsistencies at an early stage. A formal architecture model is governed by a specialised metamodel whose rules determine the consistency and completeness of the description. During development of  a system, it is quite normal for the architecture  description to be both incomplete and inconsistent.  The architects strive to produce a complete  description and to eliminate the inconsistencies. 

We describe a method of formalising the rules for the development of a new architecture, in a metamodel that the architects team can agree on, and which can co-evolve with the architecture description itself. 

By  means of an example, we show how architecture descriptions formalised in this way can aid the iterative process of architecture development and how the  model and the specialised metamodel can co-evolve. 

We then describe a design-support tool, WAVE, that we have developed to support this  process of co-evolution. This tool will calculate inconsistencies within individual and merged models. It does not insist on the architecture model always being consistent. Rather it produces reports of inconsistencies. These inconsistencies are targets for the architects to eventually remove. This means of tolerating inconsistency supports both incremental and collaborative working, essential to the development of large systems by teams of engineers.

\Section{Background}
System Architecture is an essential aspect of the design of large system. It forms the overall structure within which the components of the system interact with each other \cite{Kruchten,Maier,Rozanski,Shaw1} and consequently the basis on which architects negotiate with each other about how the system as a whole will eventually work. 

There have been many approaches to the description of System Architecture, both formal and semi-formal. We have been influenced by both, but in particular the more pragmatic methods \cite{Gokhale,Henderson2,Holt,Alloy}, in particular those that combine familiar semi-formal methods with an element of evaluation \cite{Chang,Egyed3,Shaw,Shen}. 

We are particularly concerned with methods that scale up to be applicable to very large systems \cite{Balzer,Fickas05clinicalrequirements,Henderson1,Maier,Nuseibeh01makinginconsistency}, by which we mean those that will eventually require a large team of software engineers working over an extended period of time. Such methods necessarily involve a great degree of collaboration \cite{Dekel,Nejati07matchingand,1150731}.

Large systems and collaborative development include long periods when the design is both incomplete and inconsistent. The inconsistencies arise when separate parts of the architecture description are developed independently. Many others have worked on the issue of ensuring consistency \cite{Chang2,Egyed1,Sabetzadeh_globalconsistency,Sabetzadeh07consistencychecking}, while others have addressed the issue of tolerating inconsistency \cite{Balzer,Nuseibeh01makinginconsistency}. This work has been fundamental in our development of the method and tool that we propose here.

We have also been influenced in the development of our tool, by the tools developed by others, in particular those based on a relational model of architecture  \cite{Crocopat,Crocopat2,Egyed2,Holt}. The relational algebra \cite{Date,Holt,Alloy} is an ideal formal language for giving structural architecture descriptions and goes a long way towards being appropriate for behavioural descriptions. This, we believe, is because at the level at which architecture description needs to be performed (sufficiently detailed but appropriately abstract) a relational model introduces just the right level of formality. Note that UML (and SysML) have metamodels which are described relationally. This perhaps explains why relational models are a good fit to the task of architecture description.

In the discussion at the end of the paper we mention additional areas of application, including documentation \cite{Henderson3} and modular reasoning \cite{Rushby,Haley05arguingsecurity}, both of which require architectural support. But first, we illustrate our method using familiar examples from software engineering.
%------------------------------------------------------------------------- 
\Section{A Method of Architecture Description}

\noindent System Architects, building software intensive systems, start from a mixture of user requirements, system requirements and legacy components and devise an architecture that meets the requirements while making effective use of existing components. They will describe the architecture using a mixture of diagrams and natural language that is effective as a means of communication among them and their customers 

Diagrams are most effective at indexing a description. The reader uses the diagrams to get an overview of the (part of the) architecture in which they are interested and then refers to a natural language description to learn the details. The reader will expect to find redundancy in the descriptions and consistency between related parts. For example, in the next section you will see (Figures 2 and 3) diagrams that exhibit redundancy and consistency - in this case, a class diagram and an apparently consistent sequence diagram .

\begin{table}
\begin{tabular}{|r|l|}
\hline
1. & \parbox{2.75in}{\vspace{1mm}The architects agree a preliminary metamodel, including entities, relationships and consistency rules. \vspace{1mm}}\\ 
\hline
2. & \parbox{2.75in}{\vspace{1mm}Each architect develops their part of the model, obeying as nearly as possible the current metamodel.\vspace{1mm}} \\
\hline
3. & \parbox{2.75in}{\vspace{1mm}Each architect strives to drive out inconsistencies in their part of the model. \vspace{1mm}}\\
\hline
4. & \parbox{2.75in}{\vspace{1mm}Periodically, models are merged so that cross-model inconsistencies can be eliminated. \vspace{1mm}}\\
\hline
5. & \parbox{2.75in}{\vspace{1mm}Periodically, the metamodel is evolved to encode the architects' evolving understanding of the problem domain. \vspace{1mm}}\\
\hline
\end{tabular}
\caption{Architecture Development Method}
\end{table}
An essential adjunct to the diagrams-plus-natural-language presentation are the consistency rules that the architecture will obey. A judicious use of formal language can complement these necessary aspects of presentation. So the language we choose, to describe a proposed architecture, needs to be sufficiently formal that some consistency checking can be done but not so detailed that the work of describing the architecture is as costly as building the whole system.

Hence many system architects use diagrammatic notations such as those that constitute UML and SysML and specialise them to their specific needs. This specialisation can be represented by a {\em specialised metamodel} which enumerates the entities that will be used to describe the architecture and defines the constraints that instances of these entities must obey. 

The architecture development process that we advocate is shown in Table 1. It comprises an iterative co-evolution of the architecture model and the specialised metamodel. The architects use the ``language'' defined in the metamodel to capture the architecture description. Since they work as a team, working independently on parts of the description and then merging their efforts, they will introduce inconsistencies that eventually they will strive to remove. In particular, inconsistencies arise when separately developed model-parts are merged. Sometimes, it in not the model that needs to change to eliminate inconsistencies, but the metamodel. The method of Table 1 covers all these aspects. It is of course an iterative process.

This process is supported by our tool, described in a later section, but can be carried out by a disciplined team using their normal development tools, checking the consistency manually by reading each others' contributions when models are merged. Mechanical checking of consistency requires a more formal approach, such as that supported by our tool. Before introducing that, we will develop a simple example showing an architecture model being developed collaboratively and its metamodel being evolved.
%PUT FIGURE 1 HERE
\begin{figure}
  \centering
   {\epsfig{file = mcfigure1.eps, width = 7.0 cm}}
   \caption{A simple metamodel.}
\end{figure}


\Section{An Architecture Example}

\noindent We consider a team of architects developing a large software system. The system is to be built from components that send messages to each other (probably by a mechanism such as RPC). The first thing the architectural team must determine is their metamodel. Let us assume that they are going to construct a model comprising Class diagrams (or Component diagrams) with associations between the classes recording client/server (i.e. uses) relationships. Let us also assume that they are going to record scenarios (e.g. details of Use Cases) in Sequence diagrams where messages are exchanged between components. The kind of consistencies they might wish to maintain are that messages may only pass from clients to servers and that every operation of a Component will be exercised by at least one scenario.

Examples of the Class diagrams and Sequence diagrams that might be developed are shown in Figures 2 and 3 respectively. In practice we expect such diagrams to contain many more entities than this trivial example and to be split across many separate diagrams. This is why consistency becomes an issue. The full-scale examples that we have used to evaluate our tool have contained on the order of fifty classes and comprised a few dozen independently developed diagrams.
%PUT TABLE 1 HERE
\begin{table}
\begin{tabular}{|l|l|} \hline
consistency rule & description \\ \hline
no invalid messages & \parbox{1.75in}{\vspace{1mm}For each Message from one Component to another there is a corresponding Association in some class diagram.\vspace{1mm}} \\ \hline

no untested messages & \parbox{1.75in}{\vspace{1mm}For each Association from one Component to another there is a corresponding Message in at least one scenario.\vspace{1mm}} \\ \hline

no undefined methods & \parbox{1.75in}{\vspace{1mm}For each Operation appearing on a Message there is a corresponding defintion in the receiving Component.\vspace{1mm}}  \\ \hline

no untested methods & \parbox{1.75in}{\vspace{1mm}For each Operation defined on each Component there is at least one Message in some scenario that uses it.\vspace{1mm}} \\ \hline

no cycles & \parbox{1.75in}{\vspace{1mm}For the purposes of loose-coupling, there should be no cycles in the Associations established across all Class diagrams.\vspace{1mm}} \\ \hline
\end{tabular}
\caption{Some Consistency Rules.}
\end{table}

We will use a relational model to develop the formal aspects of our architecture description as, we have discussed, many others have done before us \cite{Crocopat2,Egyed2,Holt,Alloy,Sabetzadeh07consistencychecking}. 

The metamodel that the architecture team constructs at the outset might look like that shown in Figure 1 and have the consistency rules enumerated as in Table 2.

The metamodel says that the entities appearing on the diagrams will be Components, Associations, Messages and Operations. The Components and Associations will appear respectively as boxes and arrows on Class diagrams, such as in Figures 2 and 4. The Messages and Operations will appear on Sequence diagrams, such as in Figures 3 and 5. The diagrams record relational information about these entities. For example, the Class diagrams record the fact that each Association has a source and a target, both of which are Components. Similarly, the Sequence diagrams record the fact that each Message has a sender and a receiver (both Components) as well as a method, which is a operation of the receiver. The metamodel which captures these relationships, also assumes that we have an enumeration of the Operations of each Component.


Table 2 shows an initial set of consistency rules that we assume the architecture team have enumerated. These rules are simple, but typical of the structural consistencies that the team will be trying to achieve. Basically, the rules state that every Message should be from client to server, that every Message should be tested in some scenario and that (for loose-coupling) cycles in the Class diagrams are to be avoided. We will see that, while these rules can be obeyed, the architectural team eventually chooses to relax them (and consequently evolve the metamodel).


\begin{figure}
  \centering
   {\epsfig{file = mcfigure2.eps, width = 7.0 cm}}
   \caption{Architect 1's Class Diagram.}
\end{figure}

\begin{figure}
  \centering
   {\epsfig{file = mcfigure3.eps, width = 7.0 cm}}
   \caption{Architect 1's Sequence Diagram.}
\end{figure}



\Section{Consistency }
\noindent Having agreed the metamodel and its rules, the team then begins independent development of separate parts of the architecture. Suppose Architect 1 comes up with the Class diagram in Figure 2 and the Sequence diagram in Figure 3. These two diagrams almost satisfy the five rules of Table 2. In fact, our tool notes that there is an operation of C1 which is not tested (op1), but that is the only problem. Architect 1 is satisfied with this, because it is someone else's task to exercise C1. 

In practice, when an individual architect is working on a part of an architecture, they may be dealing with a few dozen Components and developing (say) a dozen or more scenarios. Arriving at an acceptable level of internal consistency will be an iterative process. Residual inconsistencies will be (the architect hopes) resolved when their part is combined with the other parts being developed concurrently.

Architect 2 has, we imagine, concurrently developed the Class diagram shown in Figure 4 and the Sequence diagram shown in Figure 5. They receive much the same report as Architect 1 - all is consistent, except there is an operation of C2 that has not been tested (op2). 


However, when the two architects combine their models and run consistency checking again, they encounter a little difficulty.

While the inconsistencies that they knew about when developing independently are resolved by the combination of models (now, all known Operations are tested), the combination of the Class diagrams has unfortunately created a loop in the Associations between C1 and C2. This is something that will need to be resolved, either by one of the architects making modifictions to their contribution, or by a change to the metamodel.

The reason we are so concerned with inconsistency is that, for large systems, many components and many scenarios will be defined and it will considerably improve the quality of the description if these individual descriptions are eventually made consistent, while for pragmatic reasons we must tolerate inconsistency during development. 

In our example we have a choice of the remedial action to take. It may be possible for one architect to change their model. If not, the team will consider if the metamodel needs to be evolved. 

\begin{figure}
  \centering
   {\epsfig{file = mcfigure4.eps, width = 7.0 cm}}
   \caption{Architect 2's Class Diagram.}
\end{figure}
 
\begin{figure}
  \centering
   {\epsfig{file = mcfigure5.eps, width = 7.0 cm}}
   \caption{Architect 2's Sequence Diagram.}
\end{figure}
In fact, here, we have probably made the loose-coupling constraint too strong. We might allow ``local'' loops within well defined subgraphs of the whole architecture. Here for example, it would be sufficient to allow loops between Components which are of length no greater than two. If A uses B, then allowing B to use A does not really damage the coupling, we could argue (a common feature in implementations that use some form of callback).

In practice, we will wish to construct more complex relations than those that we have exemplified here and for which we will require more expressive forms than the simple relational diagrams used in this section. In the next section we describe how we have based our tooling on the relational algebra, and how this both simplifies the encoding of the consistency rules and the evolution of the metamodel.



\Section{A Design Support Tool}
\noindent The WAVE tool has been implemented to support the way of working outlined in Table 1 and illustrated in the previous section. The architects work concurrently on separate (overlapping) parts of the architecture, recording their progress in a local copy of a ``database''. In practice, we generate this database from whichever diagramming tool the architects choose to use, by dumping it as an XMI file and importing it into the WAVE tool.

In the implementation we shall describe here (available from http://ecs.soton.ac.uk/\symbol{126}ph/WAVE) the database is held as a set of relational tables and the consistency rules are implemented is scripts that compute new tables recording discovered inconsistencies. 

The use of a relational model and of scripts to define the consistency rules makes the co-evolution of the metamodel particularly adaptable and straightforward.

For example, the script that computes the first consistency rule in Table 1 is written \small
\begin{verbatim}
invalid_messages =
     diff(join(invert(sender),receiver),
          join(invert(source),target))
\end{verbatim}\normalsize
Here, {\small\verb$join$} is the relational join of two (binary) relations. The relations are those recorded in the metamodel (Figure 1) and the data in them records that displayed on the actual model diagrams. The operation {\small\verb$invert$} takes the inverse of a (binary) relation and {\small\verb$diff$} computes the set difference. So the above calculation constructs a relation which relates any two Components between which there is a Message (on some sequence diagram) but between which there is no corresponding Association on any Class diagram. This computed relation is effectively an inconsistency report - listing those places where rule 1 is disobeyed.

The fact that WAVE is scripted and stores its data in relational tables (exportable as tables or as XML) means that generating reports of inconsistencies is also straightforward. Architects can thus work with independent copies of the database and work to extend their view of the architecture and to remove inconsistencies that are reported. 

When databases are merged in WAVE the simplest, and apparently most effective, merge is simply to take the union of each copy of each relational table. WAVE has been specifically designed so that for most purposes this is the most effective merge. It does mean however that, if an entity appears in two copies of the database and is deleted by only one of the architects, it will reappear on merge. If it was required that deletes by one architect would propagate on merge, then a more elaborate (diff3-like) merge is required. Since all operations on the database are scripted, including merge, making a domain-specific merge is as straightforward as any metamodel evolution. 

We have used WAVE on a number of small projects and a couple of fairly large ones. The largest has about fifty classes (actually, components) spread over about a dozen class diagrams and another dozen sequence diagrams. In the current version of the architecture described in this largest example, WAVE lists about twenty inconsistencies that are genuine inconsistencies between model-parts developed by different architects and a smaller number that are probabably going to require the metamodel (which includes the metamodel shown here as a subset) to be co-evolved. Running the WAVE scripts against the XMI files for this architecture (dumped, as it happens, from Sparx Systems' Enterprise Architect) takes a few seconds. We have tested performance on an artificial example with around 1000 entries in each of the relational tables that are generated as intermediate structures in WAVE, to persuade ourselves that WAVE's performance will scale to realistic large-scale models. These tests have shown that WAVE can process such tables with constraints such as those listed in Table 2 in seconds rather than minutes. A more comprehensive performance analysis will be completed soon. 
 
\Section{Conclusions}
\noindent We have described a method of developing architecture descriptions based on giving a sufficiently precise metamodel that consistency can be checked during architecture development.

Rather than insist that the architecture description is kept consistent at all times, we advocate a method of iterative and cooperative development that allows the description to be periodically inconsistent.

We have shown how the metamodel can be captured formally as a relational model and argued that this method is particularly appropriate to this style of development, not least of all because it encourages the architects to embrace the whole architecture at all times and to keep in mind how far from internal consistency the description may have drifted.

The tool we have described supports this method of working, where model and metamodel are co-evolved and where inconsistency is tolerated during development. For large scale systems, where iterative and cooperative working is the norm, this tolerance of inconsistency is essential.

The method has been applied so far only to software intensive systems. It also seems appropriate to other domains. We have in hand experiments with metamodels for documentation \cite{Henderson3}, for reasoning \cite{Rushby,Haley05arguingsecurity} and have an ambition to extend the method to a broader range of systems, in particular those that have physical as well as logical structure. Ultimately our plan is to combine these domains so that describing an architecture, documenting it and reasoning about it will all be supported within the same framework.

%------------------------------------------------------------------------- 
%{\small\verb$operations(Component, Operation)$}. 

\bibliographystyle{latex8}
\bibliography{sekeconsist}

\end{document}

